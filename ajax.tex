% www-conceptos.tex
% Asignatura de SAT

\section{Ajax y tecnolog�as relacionadas}

%%%%%%%%%%%%%%%%%%%%%%%%%%%%%%%%%%%%%%%%%%%%%%%%%%%%%%%%%%%%%%%%%%%%%%

\begin{frame}
\frametitle{Algunos palabros}

{\Large
\begin{itemize}
\item DHTML
\item SPA
\item AJAX
\item Web 2.0
\item HTML5
\item Mashup
\end{itemize}
}
\end{frame}


%%%%%%%%%%%%%%%%%%%%%%%%%%%%%%%%%%%%%%%%%%%%%%%%%%%%%%%%%%%%%%%%%%%%%%

\begin{frame}
\frametitle{DHTML}

{\Large
\begin{itemize}
\item Dynamic HTML
\item T�picamente: combinaci�n de HTML, CSS (presentaci�n)
  JavaScript y DOM (manipulaci�n) 
\item La aplicaci�n corren en el cliente, pero recibe datos, y los
  env�a, del/al servidor
\item Maneja datos normalmente usando DOM
\item El estado t�picamente se mantiene en el servidor
\item Problemas debido a las diferencias de las APIs entre navegadores
\end{itemize}
}

\end{frame}

%%%%%%%%%%%%%%%%%%%%%%%%%%%%%%%%%%%%%%%%%%%%%%%%%%%%%%%%%%%%%%%%%%%%%%

\begin{frame}
\frametitle{SPA}

{\Large
\begin{itemize}
\item Single page application
\item Es un tipo de aplicaci�n DHTML (sin intervenci�n del servidor)
\item La aplicaci�n funciona completamente en el navegador
\item Maneja los datos de la propia p�gina usando DOM
\item Los datos modificados se almacenan localmente
\end{itemize}
}

\begin{flushright}
Algunos ejemplos: \\
\url{https://onepagelove.com/} \\
\url{http://www.awwwards.com/websites/single-page/} \\
\end{flushright}

\end{frame}

%%%%%%%%%%%%%%%%%%%%%%%%%%%%%%%%%%%%%%%%%%%%%%%%%%%%%%%%%%%%%%%%%%%%%%

\begin{frame}
\frametitle{Ajax}

\begin{itemize}
\item Asynchronous JavaScript and XML
\item DHTML m�s XMLHttpRequest
\item Principal cambio para el usuario: las p�ginas se actualizan, no
  se recargan completamente
\item Normalmente, intercambio de datos con el servidor v�a XML (no
  p�ginas completas HTML)
\item Muchas acciones hechas localmente (cuando no hace falta nueva
  informaci�n)
\item Ejemplo de aplicaci�n: Google Mail
\end{itemize}

\begin{flushright}
Art�culo original: \\
{ \footnotesize
\url{https://web.archive.org/web/20080702075113/http://www.adaptivepath.com/ideas/essays/archives/000385.php}
}
\end{flushright}
\end{frame}

%%%%%%%%%%%%%%%%%%%%%%%%%%%%%%%%%%%%%%%%%%%%%%%%%%%%%%%%%%%%%%%%%%%%%%

\begin{frame}
\frametitle{Web 2.0}

\begin{itemize}
\item Evoluci�n del web de colecci�n de sitios a plataforma
  inform�tica completa.
\item Proporciona aplicaciones web a usuarios finales
\item Supuestamente sustuir� a muchas aplicaciones de escritorio
\item Explotaci�n de ``efectos red'', por ejemplo con redes sociales
  (arquitectura de participaci�n)
\item Ejemplos: Google AdSense, Flickr, blogs, wikis
\item Ejemplos de tecnolog�as: CSS, XHTML, Ajax, RSS/ATOM, 
\end{itemize}

\begin{flushright}
{\small
\url{http://www.oreilly.com/pub/a/web2/archive/what-is-web-20.html}
}
\end{flushright}

\end{frame}

%%%%%%%%%%%%%%%%%%%%%%%%%%%%%%%%%%%%%%%%%%%%%%%%%%%%%%%%%%%%%%%%%%%%%%

\begin{frame}
\frametitle{HTML5}

\begin{itemize}
\item �ltima versi�n de HTML (octubre de 2014)
\item Originalmente, trabajo del WHATWG, ahora tambi�n del W3C
\item Incluye:
  \begin{itemize}
  \item Primera l�nea: \verb|<!DOCTYPE html>|
  \item Nuevo marcado HTML: $<audio>$, $<video>$, 
  \item DOM Scripting (JavaScript)
  \item APIs: Canvas 2D, drag-and-drop, off-line storage database, microdata, WebGL, SVG,...
  \end{itemize}
\item Los navegadores modernos ya lo soportan
\end{itemize}

{\small
\begin{flushright}
\url{https://developer.mozilla.org/en-US/docs/Web/Guide/HTML/HTML5} \\
\url{http://en.wikipedia.org/wiki/HTML5} \\
\url{http://www.html5rocks.com/en/} \\
\end{flushright}
}
\end{frame}


%%%%%%%%%%%%%%%%%%%%%%%%%%%%%%%%%%%%%%%%%%%%%%%%%%%%%%%%%%%%%%%%%%%%%%

\begin{frame}
\frametitle{Mashup}

\begin{itemize}
\item Combinaci�n de contenidos (y funcionalidad) de varios sitios en
  una aplicaci�n web
\item Combinaci�n usando APIs de terceros ejecutadas en servidor,
  feeds (Atom, RSS), JavaScript, etc.
\item APIs populares: eBay, Amazon, Google, Windows Live, Yahoo
\item Buena integraci�n con otros conceptos de Web 2.0
\item Puede haber mashups de mashups...
\item Idea general: creaci�n de aplicaciones mediante composici�n de
  aplicaciones web
\end{itemize}

\end{frame}

%% %%%%%%%%%%%%%%%%%%%%%%%%%%%%%%%%%%%%%%%%%%%%%%%%%%%%%%%%%%%%%%%%%%%%%%

%% \begin{frame}[fragile]
%% \frametitle{Ejemplo simple (mashup): Google gadgets}

%% \begin{itemize}
%% \item C�digo JavaScript que se baja al descargar una p�gina
%% \item Puede combinar varios servicios
%% \item Se incluye f�cilmente en HTML (mashup sin programar)
%% \item Pueden arrastrarse directamente a Google Desktop
%% \end{itemize}

%% \begin{verbatim}
%% <script src="http://gmodules.com/ig/ifr?
%% url=http://ralph.feedback.googlepages.com/googlemap.xml&
%% up_loc=Universidad%20Rey%20Juan%20Carlos,%20Mostoles,
%% %20Spain&up_zoom=Street&up_view=Hybrid&synd=open&
%% w=320&h=200&output=js">
%% </script>
%% \end{verbatim}

%% \begin{flushright}
%% \url{http://www.google.com/ig/directory?synd=open}
%% \end{flushright}

%% \end{frame}

%% %%%%%%%%%%%%%%%%%%%%%%%%%%%%%%%%%%%%%%%%%%%%%%%%%%%%%%%%%%%%%%%%%%%%%%

%% \begin{frame}
%% \frametitle{Ejemplo simple (mashup): Google gadgets (2)}

%% Implementaci�n (todo puede hacerse con una herramienta):

%% \begin{itemize}
%% \item gadget.gmanifest (XML): metainformaci�n
%% \item main.xml (XML): interfaz de usuario, objetos visibles
%% \item options.xml (XML): opciones (si las hay)
%% \item C�digo (JavaScript, VBScript): funcionalidad (corre en el
%%   navegador), usa la API de Google Desktop
%% \item Im�genes: iconos, botones, gr�ficos, etc.
%% \item Localizaci�n: ficheros con traducciones de la interfaz de usuario
%% \end{itemize}


%% \begin{flushright}
%% \url{http://desktop.google.com/script.html}
%% \end{flushright}

%% \end{frame}

%% %%%%%%%%%%%%%%%%%%%%%%%%%%%%%%%%%%%%%%%%%%%%%%%%%%%%%%%%%%%%%%%%%%%%%%

%% \begin{frame}
%% \frametitle{Ejemplo simple (mashup): Morfeo EzWeb}

%% \begin{itemize}
%% \item Django en el lado del servidor
%% \item JavaScript en el lado del cliente
%% \item Interconexi�n entre gadgets en el lado del cliente (publicaci�n
%%   / subcripci�n)
%% \item Creaci�n de aplicaciones combinando en el cliente y con
%%   servidores
%% \item Todo es software libre
%% \end{itemize}


%% \begin{flushright}
%% \url{http://ezweb.morfeo-project.org/}
%% \end{flushright}

%% \end{frame}


%% %%%%%%%%%%%%%%%%%%%%%%%%%%%%%%%%%%%%%%%%%%%%%%%%%%%%%%%%%%%%%%%%%%%%%%

%% \begin{frame}
%% \frametitle{Referencias}

%% \begin{itemize}
%% \item Web 2.0: \\
%%   \url{http://en.wikipedia.org/wiki/Web_2.0}
%% \item How To Make Your Own Web Mashup: \\
%%   \url{http://programmableweb.com/howto}
%% \end{itemize}


%% \end{frame}

