%%---------------------------------------------------------------
%%---------------------------------------------------------------

\section{Optimizaci�n}

%%---------------------------------------------------------------
%
%\begin{frame}
%\frametitle{}
%
%\begin{itemize}
%  \item 
%\end{itemize}
%
%\end{frame}


%%---------------------------------------------------------------

%\begin{frame}
%\frametitle{}
%
%\begin{center}
%\begin{figure}[p]
%\includegraphics[width=0.65\textwidth]{figs/}
%\end{figure}
%\end{center}
%
%\begin{flushright}
%{\tiny
%Source: 
%}
%\end{flushright}
%
%\end{frame}

%%---------------------------------------------------------------

\begin{frame}
\frametitle{�Por qu� optimizar?}

\begin{itemize}
  \item Se ha realizado mucho trabajo de optimizaci�n en las m�quinas de Javascript de los navegadores en los �ltimos a�os.
  \item Sin embargo, JavaScript requiere que el desarrollador realiza optimizaciones que en otros lenguajes har�a el compilador.
  \item Cada vez que se encuentra una etiqueta $<script>$, el navegador se para, descarga
el c�digo JavaScript (si fuera necesario), y se ejecuta antes de procesar el resto
de la p�gina. 
\end{itemize}

\end{frame}


%%---------------------------------------------------------------

\begin{frame}
\frametitle{Carga y ejecuci�n de JavaScript}

\begin{itemize}
  \item Pon todos los $<script>$ al final de la p�gina, antes del $</body>$. De esta manera, hay contenido que se muestra que no tiene que esperar.
  \item Agrupa los scripts. Cuantos menos $<script>$s tiene una p�gina, antes cargar� y se volver� interactiva. 
  \item Utiliza t�cnicas para descargar JavaScript de manera no bloqueante:
  \begin{itemize}
    \item Utilizando el atributo \texttt{defer} de $<script>$ (s�lo IE y Firefox 3.5+)
    \item Crea elementos $<script>$ din�micos
    \item Descarga c�digo JavaScript utilizando AJAX y luego inyecta el c�digo en la p�gina.
  \end{itemize}
  \item Maximiza el uso de la cach� (y CDN) para scripts en JavaScript.
\end{itemize}


\end{frame}

%%---------------------------------------------------------------

\begin{frame}
\frametitle{Trabajando con el DOM}

\begin{itemize}
  \item El acceso al DOM es caro. Minimiza el acceso al DOM. Intenta trabajar al m�ximo en JavaScript.
  \item Realiza todos los cambios en una �nica vez. Tambi�n (o especialmente) con los cambios de estilos.
  \item Para accesos repetidos al DOM, utiliza variables locales.
  \item Utiliza \texttt{document.getElementById} o el identificador en JQuery
\end{itemize}

\end{frame}


%%---------------------------------------------------------------

\begin{frame}
\frametitle{Algoritmos y control de flujo}

\begin{itemize}
  \item Los bucles \texttt{for}, \texttt{while} y \texttt{do-while} tienen caracter�sticas de potencia similares y ninguno es significativamente m�s r�pido que los dem�s.
  \item Se han de evitar los bucles \texttt{for-in}, a menos de que desconozcamos el n�mero de propiedades de un objeto.
  \item En general, \texttt{switch} es m�s eficiente que \texttt{if-else}, aunque no siempre es la mejor soluci�n.
\end{itemize}

\end{frame}




%%---------------------------------------------------------------

\begin{frame}
\frametitle{Manteniendo la responsividad del interfaz}

\begin{itemize}
  \item Ning�n c�digo de JavaScript es tan importante como para afectar negativamente la experiencia de usuario
  \item Ning�n script de JavaScript deber�a llevar m�s de 20 segundos.
  \item Utiliza \texttt{Web workers} para ejecutar c�digo JavaScript fuera el hilo de interfaz de usuario, evitando as� su bloqueo
\end{itemize}

\end{frame}


%%---------------------------------------------------------------

\begin{frame}
\frametitle{Herramientas para optimizar}

\begin{itemize}
  \item Firebug Profiler: recoge informaci�n de cada funci�n de JavaScript y de cu�nto tiempo ha llevado su ejecuci�n
  \item JSLint: \url{http://www.jslint.com/lint.html}
  \item Closure Tools: \url{https://developers.google.com/closure/compiler/}
  \item YUI Compressor: \url{http://yui.github.io/yuicompressor/}
\end{itemize}

\end{frame}


