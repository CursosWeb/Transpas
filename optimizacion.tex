%%---------------------------------------------------------------
%%---------------------------------------------------------------

\section{Optimización}

%%---------------------------------------------------------------
%
%\begin{frame}
%\frametitle{}
%
%\begin{itemize}
%  \item 
%\end{itemize}
%
%\end{frame}


%%---------------------------------------------------------------

%\begin{frame}
%\frametitle{}
%
%\begin{center}
%\begin{figure}[p]
%\includegraphics[width=0.65\textwidth]{figs/}
%\end{figure}
%\end{center}
%
%\begin{flushright}
%{\tiny
%Source: 
%}
%\end{flushright}
%
%\end{frame}

%%---------------------------------------------------------------

\begin{frame}
\frametitle{¿Por qué optimizar?}

\begin{itemize}
  \item Se ha realizado mucho trabajo de optimización en las máquinas de Javascript de los navegadores en los últimos años.
  \item Sin embargo, JavaScript requiere que el desarrollador realiza optimizaciones que en otros lenguajes haría el compilador.
  \item Cada vez que se encuentra una etiqueta $<script>$, el navegador se para, descarga
el código JavaScript (si fuera necesario), y se ejecuta antes de procesar el resto
de la página. 
\end{itemize}

\end{frame}


%%---------------------------------------------------------------

\begin{frame}
\frametitle{Carga y ejecución de JavaScript}

\begin{itemize}
  \item Pon todos los $<script>$ al final de la página, antes del $</body>$. De esta manera, hay contenido que se muestra que no tiene que esperar.
  \item Agrupa los scripts. Cuantos menos $<script>$s tiene una página, antes cargará y se volverá interactiva. 
  \item Utiliza técnicas para descargar JavaScript de manera no bloqueante:
  \begin{itemize}
    \item Utilizando el atributo \texttt{defer} de $<script>$ (sólo IE y Firefox 3.5+)
    \item Crea elementos $<script>$ dinámicos
    \item Descarga código JavaScript utilizando AJAX y luego inyecta el código en la página.
  \end{itemize}
  \item Maximiza el uso de la caché (y CDN) para scripts en JavaScript.
\end{itemize}


\end{frame}

%%---------------------------------------------------------------

\begin{frame}
\frametitle{Trabajando con el DOM}

\begin{itemize}
  \item El acceso al DOM es caro. Minimiza el acceso al DOM. Intenta trabajar al máximo en JavaScript.
  \item Realiza todos los cambios en una única vez. También (o especialmente) con los cambios de estilos.
  \item Para accesos repetidos al DOM, utiliza variables locales.
  \item Utiliza \texttt{document.getElementById} o el identificador en JQuery
\end{itemize}

\end{frame}


%%---------------------------------------------------------------

\begin{frame}
\frametitle{Algoritmos y control de flujo}

\begin{itemize}
  \item Los bucles \texttt{for}, \texttt{while} y \texttt{do-while} tienen características de potencia similares y ninguno es significativamente más rápido que los demás.
  \item Se han de evitar los bucles \texttt{for-in}, a menos de que desconozcamos el número de propiedades de un objeto.
  \item En general, \texttt{switch} es más eficiente que \texttt{if-else}, aunque no siempre es la mejor solución.
\end{itemize}

\end{frame}




%%---------------------------------------------------------------

\begin{frame}
\frametitle{Manteniendo la responsividad del interfaz}

\begin{itemize}
  \item Ningún código de JavaScript es tan importante como para afectar negativamente la experiencia de usuario
  \item Ningún script de JavaScript debería llevar más de 20 segundos.
  \item Utiliza \texttt{Web workers} para ejecutar código JavaScript fuera el hilo de interfaz de usuario, evitando así su bloqueo
\end{itemize}

\end{frame}


%%---------------------------------------------------------------

\begin{frame}
\frametitle{Herramientas para optimizar}

\begin{itemize}
  \item Firebug Profiler: recoge información de cada función de JavaScript y de cuánto tiempo ha llevado su ejecución
  \item JSLint: \url{http://www.jslint.com/lint.html}
  \item Closure Tools: \url{https://developers.google.com/closure/compiler/}
  \item YUI Compressor: \url{http://yui.github.io/yuicompressor/}
\end{itemize}

\end{frame}


