% $Id$
%

\section{Hojas de estilo CSS3}


%%---------------------------------------------------------------

\begin{frame}
\frametitle{�Qu� es CSS3?}

\begin{itemize}
  \item CSS3 ofrece una gran variedad de maneras nuevas para el dise�o de hojas de estilo
  \item Al ser tan amplio el desarrollo de CSS3, se ha dividido en m�dulos:
  \begin{itemize}
    \item El modelo de caja
    \item Listas
    \item Presentaci�n de hiperv�culos
    \item Voz
    \item Fondos y bordes
    \item Efectos de texto
    \item ...
  \end{itemize}
\end{itemize}

\end{frame}


%%---------------------------------------------------------------

\begin{frame}
\frametitle{CSS3 todav�a en desarrollo}

\begin{itemize}
  \item Hay m�dulos, la mayor�a, cuya especificaci�n no est� terminada
  \item Hay m�dulos terminados, pero los navegadores no los implementan
  \item Es muy importante conocer el estado de la implementaci�n en navegadores
  \item Hay navegadores que tienen reglas espec�ficas (temporales)
  \begin{itemize}
    \item Comprobar si est� implementado en \url{http://caniuse.com/}
  \end{itemize}
\end{itemize}

\end{frame}


%%---------------------------------------------------------------

\begin{frame}
\frametitle{�Qu� hay de nuevo en CSS3?}

\begin{itemize}
  \item La buena noticia son que CSS3
  \begin{itemize}
    \item es {\bf compatible hacia atr�s}
    \item Mantiene la {\bf misma filosof�a}
  \end{itemize}
  \item B�sicamente, CSS3 a�ade nuevos selectores y propiedades
  \item Algunas son funcionalidades nuevas (animaciones o gradientes),
y otras permiten dise�os m�s sencillos (p.ej. el uso de columnas)
\end{itemize}

\end{frame}


%%---------------------------------------------------------------

\begin{frame}[fragile]
\frametitle{Especificidades de navegadores}

P.ej. border-radius:

\begin{footnotesize}
\begin{verbatim}
.box {
  -moz-border-radius: 10px;
  -webkit-border-radius: 10px;
  border-radius: 10px;
}
\end{verbatim}
\end{footnotesize}

(�imaginaos lo que era antes conseguir bordes redondeados!)

\end{frame}


%%---------------------------------------------------------------

\begin{frame}[fragile]
\frametitle{M�dulo Selectores}

\begin{itemize}
  \item first-child: primer elemento de la etiqueta padre
  \item last-child: �ltimo elemento de la etiqueta padre
  \item nth-child: selecciona m�ltiples elementos seg�n su posici�n en el �rbol
\end{itemize}

\begin{footnotesize}
\begin{verbatim}
p:first-child {
    background-color: yellow;
}
box:last-child {
    padding: 12px;
} 
\end{verbatim}
\end{footnotesize}

\end{frame}


%%---------------------------------------------------------------

\begin{frame}[fragile]
\frametitle{M�dulo Colores}

\begin{itemize}
  \item opacity: indica la opacidad de un elemento
  \item rgba: indica la opacidad de un color con el 'alpha'
\end{itemize}

\begin{footnotesize}
\begin{verbatim}
  /* red with opacity */
 .box1 {background-color: rgba(255,0,0,0.3);}  
  /* green with opacity */
 .box2 {background-color: rgba(0,255,0,0.3);}  
  /* blue with opacity */
 .box3 {background-color: rgba(0,0,255,0.3);}  
\end{verbatim}
\end{footnotesize}

\end{frame}


%%---------------------------------------------------------------

\begin{frame}[fragile]
\frametitle{M�dulo Fuentes}

\begin{itemize}
  \item @font-face: permite utilizar fuentes no instaladas en el cliente
\end{itemize}

\begin{footnotesize}
\begin{verbatim}
@font-face {
  font-family: "Essays 1743";
  font-weight: bold;
  src: url("f/Essays1743-bold.woff") format("woff");
}
\end{verbatim}
\end{footnotesize}

\end{frame}


%%---------------------------------------------------------------

\begin{frame}
\frametitle{M�dulo Fondos}

\begin{itemize}
  \item Se puede a�adir m�s de una imagen como fondo
  \item border-radius: se pueden a�adir bordes redondeados
  \item box-shadow: incluye sombras en cajas
\end{itemize}

\end{frame}

% 1 - CSS2
% 2 - bordes redondeados
% 3 - text shadow abbr
% 4 - shadow
% 5 - transparente
% 6 - gradiente
% 7 - rgba en los bordes
% 8 - rotate (a�adiendo algo de margen)
% 9 - scale
