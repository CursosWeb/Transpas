% $Id$
%

\section{Bootstrap}


%%---------------------------------------------------------------

\begin{frame}
\frametitle{�Qu� es Bootstrap?}

\begin{itemize}
  \item Bootstrap es un framework libre para desarrollo web
  \item Realizado por ingenieros de Twitter
  \item Incluye plantillas HTML y CSS con tipograf�as, formas, botones, cuadros, barras de navegaci�n, carruseles de im�genes y muchas otras
  \item Tambi�n existe la posibilidad de utilizar plugins de JavaScript
  \item Aunque su preferencia es \emph{mobile first}, permite crear dise�os que se ven bien en m�ltiples dispositivos (\emph{responsive design})
\end{itemize}

\end{frame}

%%---------------------------------------------------------------

\begin{frame}
\frametitle{Ventajas de Bootstrap (para ingenieros)}

\begin{itemize}
  \item Es sencillo y r�pido
  \item Se adapta a distintos dispositivos (\emph{responsive design})
  \item Proporciona un dise�o consistente
  \item Es compatible con los navegadores modernos
  \item Es software libre
\end{itemize}

\end{frame}

%%---------------------------------------------------------------

\begin{frame}[fragile]
\frametitle{Ficheros de Bootstrap}

\begin{footnotesize}
\begin{verbatim}
bootstrap/
|--- css/
|   |--- bootstrap.css
|   |--- bootstrap.css.map
|   |--- bootstrap.min.css
|   |--- bootstrap-theme.css
|   |--- bootstrap-theme.css.map
|   |--- bootstrap-theme.min.css
|--- js/
|   |--- bootstrap.js
|   |--- bootstrap.min.js
|--- fonts/
    |--- glyphicons-halflings-regular.eot
    |--- glyphicons-halflings-regular.svg
    |--- glyphicons-halflings-regular.ttf
    |--- glyphicons-halflings-regular.woff
    |--- glyphicons-halflings-regular.woff2
\end{verbatim}
\end{footnotesize}

\end{frame}


%%---------------------------------------------------------------

\begin{frame}[fragile]
\frametitle{La plantilla de Bootstrap}


\begin{footnotesize}
\begin{verbatim}
<!DOCTYPE html>
<html>
<head>
  <meta charset="utf-8">
  <title>Basic Bootstrap Template</title>
  <meta name="viewport" content="width=device-width, initial-scale=1.0">
  <link rel="stylesheet" type="text/css" href="css/bootstrap.min.css">
</head>
<body>
  <h1>Hello, world!</h1>
  <script src="http://code.jquery.com/jquery.min.js"></script>
  <script src="js/bootstrap.min.js"></script>
</body>
</html>
\end{verbatim}
\end{footnotesize}

\end{frame}



%%---------------------------------------------------------------

\begin{frame}[fragile]
\frametitle{Bootstrap en CDN}

\begin{footnotesize}
\begin{verbatim}
 <!-- Latest compiled and minified CSS -->
<link rel="stylesheet" 
href="http://maxcdn.bootstrapcdn.com/bootstrap/3.2.0/css/bootstrap.min.css">

<!-- jQuery library -->
<script 
 src="https://ajax.googleapis.com/ajax/libs/jquery/1.11.1/jquery.min.js">
</script>

<!-- Latest compiled JavaScript -->
<script 
 src="http://maxcdn.bootstrapcdn.com/bootstrap/3.2.0/js/bootstrap.min.js">
</script>
\end{verbatim}
\end{footnotesize}

Con un CDN (Content Delivery Network) no hace falta tener Bootstrap en nuestros
archivos. Adem�s, si un usuario ya ha descargado esas URLs, probablemente
las tenga ya en la cach� del navegador (con el consiguiente ahorro de tiempo).

\end{frame}

