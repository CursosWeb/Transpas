\section{WebAssembly}

% Transparencia 1: ¿Qué es WebAssembly?
\begin{frame}{¿Qué es WebAssembly?}
    \begin{itemize}
        \item \textbf{Definición:} Formato de instrucciones binarias para una máquina virtual basada en stack.
        \item \textbf{Objetivo:} Ejecutar código a velocidad cercana a nativa ($near-native\ speed$) en el navegador.
        \item \textbf{Complemento, no sustituto:} Diseñado para trabajar junto a JavaScript, no para reemplazarlo.
        \item \textbf{Portabilidad:} Formato abierto y estándar del W3C compatible con todos los navegadores modernos.
    \end{itemize}
\end{frame}

% Transparencia 1: ¿Qué es WebAssembly?
\begin{frame}{WebAssembly Visualmente}

\begin{figure}[h] % [h] sugiere que se coloque "aquí" (here)
    \centering
    \includegraphics[width=10cm]{figs/wasm.png}
    \caption{Diagrama de arquitectura WebAssembly} % Opcional
    \label{fig:wasm_diagram} % Opcional para hacer referencias
\end{figure}

\end{frame}

% Transparencia 2: ¿Cómo funciona? El Pipeline
\begin{frame}{El flujo de trabajo de Wasm}
    WebAssembly permite que lenguajes de alto nivel se ejecuten en la web:
    \vspace{0.3cm}
    \begin{enumerate}
        \item \textbf{Compilación:} Lenguajes como C++, Rust o Go se compilan a un archivo \texttt{.wasm}.
        \item \textbf{Carga:} El navegador descarga el binario (mucho más ligero que un JS equivalente).
        \item \textbf{Validación y Ejecución:} Se ejecuta en un entorno de \textit{sandbox} seguro.
    \end{enumerate}
    
\end{frame}

% Transparencia 3: Ventajas Clave
\begin{frame}{¿Por qué es revolucionario?}
    \begin{block}{Rendimiento}
        Decodificación y ejecución mucho más rápida que el parseo de JavaScript.
    \end{block}
    \begin{block}{Seguridad}
        Ejecución en un entorno aislado (\textit{sandboxed}) que respeta las políticas de seguridad del navegador.
    \end{block}
    \begin{block}{Ecosistema}
        Permite reutilizar bibliotecas complejas (FFmpeg, AutoCAD, motores de juegos como Unity) directamente en la web.
    \end{block}
\end{frame}

% Transparencia 4: Tendencia 2026 - Wasm fuera de la Web
\begin{frame}{El futuro: Wasm + WASI}
    En 2026, Wasm ha saltado del navegador al servidor:
    \begin{itemize}
        \item \textbf{WASI (WebAssembly System Interface):} Una interfaz estándar para que Wasm interactúe con el sistema operativo (archivos, red).
        \item \textbf{Microservicios:} Alternativa a Docker; más ligero, inicio en milisegundos y mayor densidad de cómputo.
        \item \textbf{Edge Computing:} Ejecución de lógica cerca del usuario en redes CDN de forma ultrarrápida.
    \end{itemize}
    
\end{frame}

