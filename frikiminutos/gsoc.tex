%
%

%%-----------------------------------------------------
%%-----------------------------------------------------
\section{Lo importante es participar}

%%-----------------------------------------------------
\begin{frame}
\frametitle{Google Summer of Code}

\begin{columns}[T]
\begin{column}{.38\textwidth}
\includegraphics[width=6.5cm]{figs/gsoc-logo}

\begin{flushright}
{\small
\url{https://developers.google.com/open-source/soc/}
}
\end{flushright}

\end{column}%
\hfill%
\begin{column}{.60\textwidth}
{\Large
\begin{itemize}
\item Estudiantes post-secundaria
\item Mayores 18 años
\item Beca de tres meses (5.500 USD en 2015) 
\item Desarrollo para proyectos de software libre
\item Mentores en los proyectos
\item Dos selecciones: proyectos y becarios
\item Desde 2005
\end{itemize}
}
\end{column}%
\end{columns}

\end{frame}

%%-----------------------------------------------------
\begin{frame}
\frametitle{¿Quieres participar?}

{\Large

\begin{itemize}
\item Lee la documentación (empieza por las FAQ)
\item Mira ejemplos de otros años (hay muchos)
\item Elige tu proyecto, y tu idea de colaboración \\
  (comienza con las ideas propuestas)
\item Discute tu idea con el mentor potencial
\item Envía tu solicitud
\item Envía más detalles si te los piden
\end{itemize}

\begin{center}
¡Suerte!
\end{center}
}

\end{frame}

%%-----------------------------------------------------
\begin{frame}
\frametitle{¿Y qué gano si participo?}

{\Large

\begin{itemize}
\item Una buena tarjeta de visita \\
  Ser uno de los algo más de 1.000 GSOC anuales
\item La beca que te paga Google
\item Trabajar con proyectos reales en código real
\item Quizás, que incorporen tu código al proyecto
\item Conocer a tu mentor, y a otros desarrolladores
\end{itemize}

\begin{center}
Trabajar mucho, pasártelo bien
\end{center}
}

\end{frame}


