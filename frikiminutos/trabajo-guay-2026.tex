\section{Trabajo top 2026}

%-----------------------    ---------------------------------

% Transparencia 1: Del "Coding" al "Prompting & Verification"
\begin{frame}{Evolución de la Prueba Técnica}
En 2026, se asume que usarás IA generativa. El enfoque ha cambiado:
\begin{itemize}
    \item \textbf{Ingeniería de Prompts Técnicos:} Se evalúa cómo descompones un problema complejo en instrucciones para la IA. ¿Sabes guiar a la herramienta para evitar errores (alucinaciones)?
    \item \textbf{Revisión Crítica:} El entrevistador puede darte un código generado por IA con un error lógico sutil. Tu trabajo es encontrarlo, explicar por qué falló y corregirlo.
    \item \textbf{Conocimiento Interno:} No basta con que la IA lo escriba. Debes explicar la complejidad espacial, por qué esa estructura de datos es la óptima, o la arquitectura del sistema.
\end{itemize}

\end{frame}

% Transparencia 2: Sistemas Híbridos y Conectividad
\begin{frame}{El nuevo Stack del Ingeniero Junior}
Se buscan conocimientos que la IA no puede ``improvisar'' sin contexto real:
\begin{itemize}
    \item \textbf{Integración de LLMs:} ¿Cómo conectarías una API de IA con un sistema de backend? Conceptos de latencia, tokens y costes.
    \item \textbf{Últimas tecnologías:} Demuestra que has aprendido a usar las últimas herramientas generativas en los últimos seis meses. La curiosidad tecnológica es el filtro principal para Juniors.
    \item \textbf{Trabajo colaborativo:} Se espera entiendas el flujo de trabajo colaborativo (incluye idiomas) y conozcas tecnologías y métodos colaborativos (git: trabajar en ramas, hacer Pull Requests...).
    \item \textbf{Experiencia:} Muestra tu perfil en GitHub con proyectos propios y actividad en otros repositorios.
\end{itemize}

\end{frame}

% Transparencia Soft Skills en la era de la Automatización
\begin{frame}{Diferenciación: Human-Centric Skills}
Cuanto más fácil es generar código, más valioso es el criterio humano:
\begin{itemize}
    \item \textbf{Comunicación Eficaz:} Capacidad de explicar la arquitectura del sistema a un equipo multidisciplinar.
    \item \textbf{Pensamiento de Producto:} La IA hace el ``cómo'', tú decides el ``qué''. ¿Por qué esta funcionalidad es importante para el usuario?
    \item \textbf{Capacidad de Aprendizaje:} Las Big Tech valoran la ``velocidad de rampa''. Demuestra que puedes aprender rápidamente.
    \item \textbf{Ética y Sesgos:} Se preguntará sobre la responsabilidad de los algoritmos. ¿Cómo aseguras que tu software no es sesgado?

\end{itemize}
\end{frame}
