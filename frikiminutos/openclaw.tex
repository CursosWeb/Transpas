\section{OpenClaw}


%-----------------------    ---------------------------------

\begin{frame}
\frametitle{El agente de IA ya está aquí (o quizás no)}

\begin{center}
  \includegraphics[width=10cm]{figs/openclaw.png}
\end{center}


\begin{center}
{\LARGE
https://openclaw.ai/
}
\end{center}

\end{frame}

%-----------------------    ---------------------------------


% Transparencia 2: Funcionalidad y Arquitectura
\begin{frame}{¿Qué es OpenClaw?}
    \begin{block}{Naturaleza del Software}
        Es un agente de IA autónomo diseñado para ejecutarse localmente y ejecutar tareas complejas a través de interfaces de mensajería.
    \end{block}
    \begin{itemize}
        \item \textbf{Integración:} Funciona mediante plataformas como WhatsApp, Telegram y Signal.
        \item \textbf{Tecnología:} Programado principalmente en \textbf{TypeScript} y distribuido bajo la \textbf{licencia MIT}.
        \item \textbf{Capacidades:} Automatización de flujos de trabajo (workflows), ejecución de comandos en el sistema y persistencia de memoria local.
    \end{itemize}
\end{frame}

% Transparencia 1: Origen y Evolución
\begin{frame}{Historia y Evolución de OpenClaw}
    \begin{itemize}
        \item \textbf{Lanzamiento Original:} Desarrollado por Peter Steinberger a finales de 2025 bajo el nombre de \alert{Clawdbot}.
        \item \textbf{Crecimiento Viral:} Alcanzó las 100,000 estrellas en GitHub en apenas dos meses, convirtiéndose en uno de los repositorios de mayor crecimiento.
        \item \textbf{Cambios de Marca (Rebranding):}
        \begin{itemize}
            \item \textbf{Moltbot:} Renombrado tras una disputa de marca con Anthropic (por su marca \textit{Claude}).
            \item \textbf{OpenClaw:} Nombre definitivo adoptado a principios de 2026 para reflejar su naturaleza abierta y comunitaria.
        \end{itemize}
    \end{itemize}
\end{frame}


% Transparencia 3: Impacto y Controversias
\begin{frame}{Recepción e Impacto en el Ecosistema}
    \begin{itemize}
        \item \textbf{Moltbook:} El éxito de OpenClaw impulsó la creación de \textit{Moltbook}, una red social diseñada específicamente para que interactúen agentes de IA.
        \item \textbf{Seguridad y Privacidad:} 
        \begin{itemize}
            \item Al ejecutarse localmente, ofrece mayor privacidad que las IAs en la nube.
            \item Sin embargo, expertos advierten sobre los riesgos de dar acceso elevado a agentes autónomos.
        \end{itemize}
        \item \textbf{Soberanía Tecnológica:} Destacado por medios como \textit{Wired} y \textit{Forbes} como un cambio hacia sistemas que actúan de forma independiente.
    \end{itemize}
\end{frame}
