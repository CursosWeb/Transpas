%
%

%%-----------------------------------------------------
%%-----------------------------------------------------
\section{Localizando a quien se deje}

%%-----------------------------------------------------
\begin{frame}
\frametitle{Escenario}

{\Large
Queremos saber quien está en nuestro edificio:

\begin{itemize}
\item Con el mínimo esfuerzo nuestro posible.
\item Con el mínimo esfuerzo por parte de quienes están en el edificio.
\itme Pero podemos suponer una colaboración por su parte \\
(están interesados en que se sepa que están).
\item El edificio no es muy grande, y está aislado.
\item Una solución aproximada es suficiente.
\end{itemize}
}

{\huge
\begin{center}
¿Ideas?
\end{center}
}
\end{frame}

%%-----------------------------------------------------
\begin{frame}
\frametitle{¿Y si usamos WiFi?}

{\Large
\begin{itemize}
\item Casi todos llevan teléfono
\item Casi todos llevan WiFi activado
\item Cada teléfono usa una MAC WiFi distinta
\item Podemos pedir un registro de MACs (app web simple)
\end{itemize}
}

\vspace{1cm}

{\huge
\begin{center}
¿Cómo sabemos quién está en el edificio?
\end{center}
}

\end{frame}

%%-----------------------------------------------------
\begin{frame}
\frametitle{Detectando MACs en nuestra red WiFi}

{\Large

\begin{itemize}
\item Si somos el punto de acceso (AP), sabemos todas las MAC conectadas
\item Si escuchamos en un canal, recibimos todas las MAC que emiten
\item Pero la electrónica y el software tienen que permitirlo
\end{itemize}
}

El caso de Android:

\begin{itemize}
\item Si tenemos acceso root (eg, CyanogenMod), tenemos un kernel Linux.
\item La electrónica y el software permiten modo AP.
\item Podemos ver todo lo que ve el kernel
\item De hecho, para muchas cosas no hace falta estar en modo AP.
\end{itemize}

\begin{flushright}
{\small
  \url{https://github.com/rorist/android-network-discovery}
}
\end{flushright}
\end{frame}
