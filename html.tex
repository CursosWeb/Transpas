% $Id$
%


\section{HTML: HyperText Markup Language}

%%---------------------------------------------------------------

\begin{frame}[fragile]
\frametitle{Basic structure of a document}

\begin{verbatim}
<!DOCTYPE html>
<html>
 <head>
  <title> Tittle </title>
 </head>
 <body>
  <p>This is a paragraph</p>
 </body>
</html>
\end{verbatim}

\textbf{Ejercicio:} Escribe una página HTML simple, y mírala en el navegador.

\end{frame}


%%---------------------------------------------------------------

\begin{frame}[fragile]
\frametitle{Basic structure of a document (2)}

[Nota: en general, se describirá HTML5]

\begin{itemize}
\item \verb|<!DOCTYPE html>| es la marca de HTML5
\item En HTML 4.x era más complicado...
\item En general, cada elemento se abre y se cierra
\item La estructura de un documento es un árbol \\
  (cada elemento va dentro de otro)
\item Elemento raíz: HTML
\item Elementos bajo HTML: HEAD (indicaciones) y BODY (contenidos) 
\end{itemize}

\end{frame}

%%---------------------------------------------------------------

\begin{frame}[fragile]
\frametitle{Sintaxis básica}

Sintaxis similar a la de XML:

\begin{itemize}
\item Etiquetas (elementos, marcas): 
\begin{verbatim}
<p> ... </p> 
<p/>
\end{verbatim}

\item Atributos: 
\begin{verbatim}
<p id="abstract">
\end{verbatim}
\item Las etiquetas han de estar ``encapsuladas'': \\
  se cierran en orden inverso al que se abrieron \\
  (esto es, van siempre ``unas dentro de otras'', son contenedores)
\item Las etiquetas son nodos en el árbol, \\
  los atributos anotaciones de los nodos
\item Caracteres ``escapados'': \verb|&lt;| ($<$) \verb|&gt;| ($>$)
\end{itemize}

\textbf{Ejercicio:} Escribe una página HTML con cabecera (que tenga al menos un título) y cuerpo, con las dos sintaxis de etiquetas, y con elementos que tengan atributos.

\end{frame}

%%---------------------------------------------------------------

\begin{frame}[fragile]
\frametitle{Elemento HTML}

\begin{itemize}
\item Sintaxis básica:

\begin{verbatim}
<html lang="es">
\end{verbatim}

\item ``lang'' es el idioma (primario) del texto

\item Contiene un elemento HEAD y un elemento BODY
\end{itemize}

\begin{flushright}
Language tags in HTML and XML: \\
\url{http://www.w3.org/International/articles/language-tags/} \\
\end{flushright}
\end{frame}

%%---------------------------------------------------------------

\begin{frame}[fragile]
\frametitle{Elemento HEAD}

Información para el navegador y para bots.

Ejemplo:

\begin{verbatim}
<head>
  <meta charset="utf-8" />
  <title>El titulo</title>
  <link rel="stylesheet" type="text/css" href="main.css" />
  <link rel="alternate" type="application/atom+xml"
                        title="Canal RSS"
                        href="canal.rss" />
  <link rel="shortcut icon" href="/favicon.ico" />
</head>
\end{verbatim}

\end{frame}

\begin{frame}[fragile]
\frametitle{Elemento HEAD (2)}

\begin{itemize}
\item META:
  \begin{itemize}
  \item juego de caracteres \\
    (simplifica versiones pre-HTML5, \verb|http-equiv|)
  \item  description
  \end{itemize}
\item LINK puede apuntar a complementos para la página
  \begin{itemize}
  \item rel="stylesheet": hoja de estilo CSS
  \end{itemize} 
\item LINK puede apuntar a otros recursos relacionados
  \begin{itemize}
  \item \verb|rel="alternate"|: contenido equivalente de otros tipos (type) \\
    o en otros idiomas (hreflang)
  \item \verb|rel="author"|: autor
  \item \verb|rel="next"|, \verb|rel="prev"|: anterior, posterior
  \item \verb|rel="shortcut icon"|: icono de la página
  \item Otros: license, nofollow, search, tag, etc.
  \end{itemize} 
\item Otros: STYLE, SCRIPT (CSS o JavaScript embebidos)
\end{itemize}

\end{frame}

%%---------------------------------------------------------------

\begin{frame}[fragile]
\frametitle{Elementos en BODY}

\begin{itemize}
\item H1 - H6: cabeceras (headings)
\item P: parrafos de texto
\item A: ancla (anchor) (absoluto a url, absoluto a recurso, relativo)
\begin{verbatim}
<a href="http://linkedsite/url.html">Documento</a>
<a href="/url.html">Documento en mismo sitio</a>
<a href="url.html">Documento en mismo sitio y "dir"</a>
\end{verbatim}

\item UL, OL, DL: listas (sin ordenar, ordenadas, de definiciones)
\begin{verbatim}
<ul>
  <li>Un elemento</li>
  <li>Otro elemento</li>
</ul>
\end{verbatim}

\end{itemize}

\end{frame}

%%---------------------------------------------------------------

\begin{frame}[fragile]
\frametitle{Elementos en BODY (2)}

\begin{itemize}
\item Tabla
\begin{verbatim}
<table>
 <tr>
  <td>Primera fila, primera columna</td>
  <td>Primera fila, segunda columna</td>
 </tr>
 <tr>
  <td>Segunda fila, primera columna</td>
  <td>Segunda fila, segunda columna</td>
 </tr>
</table> 
\end{verbatim}

\end{itemize}

\end{frame}

%%---------------------------------------------------------------

\begin{frame}[fragile]
\frametitle{Elementos en BODY (3)}

\begin{itemize}
\item DIV: contenedor genérico (referncias para CSS)
\item HEADER, FOOTER, NAV, ARTICLE, SECTION, ASIDE: \\
  elementos de sección
\item IMG: imágenes
\begin{verbatim}
<img src="gsyc-bg.png" alt="Logo de GSyC">
<img src="gsyc-bg.png" alt="Logo de GSyC" width="300" height="240">
\end{verbatim}

\item MAP, AREA: mapa de imagen en el lado del cliente
\end{itemize}

\end{frame}

%%---------------------------------------------------------------

\begin{frame}[fragile]
\frametitle{Elementos en BODY (4)}

\begin{itemize}
\item FORM, FIELDSET, LABEL, INPUT

\begin{verbatim}
<form action="recurso" method="post">
  <fieldset>
    <legend>Formulario</legend>
 
    <label>Nombre</label>
    <input type="text" name="nombre"><br />
 
    <label>Apellido</label>
    <input type="text" name="apellido"><br />
 
    <input type="submit" name="persona">
 </fieldset>
</form>
\end{verbatim}

\end{itemize}
\end{frame}

%%---------------------------------------------------------------

\begin{frame}
\frametitle{Material complementario}

\begin{itemize}
\item HyperText Markup Language (Wikibook): \\
  \url{http://en.wikibooks.org/wiki/HTML_Programming}
\item HTML5: A tutorial for beginners: \\
  \url{http://www.html-5-tutorial.com/}
\item Dive into HTML5: \\
  \url{http://diveintohtml5.info}
\item HTML5 (Wikipedia): \\
  \url{http://en.wikipedia.org/wiki/HTML5}
\item Web Fundametals (Code Academy): \\
  \url{http://www.codecademy.com/tracks/web}
\end{itemize}


\end{frame}

