%
% $Id: $
%
%
% Compilar a .pdf con LaTeX (pdflatex)
% Es necesario instalar Beamer (paquete latex-beamer en Debian)
%

%
% Gr�ficos:
% Los gr�ficos pueden suministrarse en PNG, JPG, TIF, PDF, MPS
% Los EPS deben convertirse a PDF (usar epstopdf)
%

\documentclass{beamer}
\usetheme{Warsaw}
%\usebackgroundtemplate{\includegraphics[width=\paperwidth]{format/libresoft-bg-soft.png}}
\usepackage[spanish]{babel}
\usepackage[latin1]{inputenc}
\usepackage{graphics}
\usepackage{amssymb} % Simbolos matematicos

%\definecolor{libresoftgreen}{RGB}{162,190,43}
%\definecolor{libresoftblue}{RGB}{0,98,143}

%\setbeamercolor{titlelike}{bg=libresoftgreen}

%% Metadatos del PDF.
\hypersetup{
  pdftitle={Introducci�n pr�ctica a las tecnolog�as web},
  pdfauthor={Jes�s M. Gonz�lez Barahona},
  pdfcreator={GSyC/LibreSoft, Universidad Rey Juan Carlos},
  pdfproducer=PDFLaTeX,
  pdfsubject={Jornada de acogida a alumnos de ESO, Fuenlabrada 2015},
}
%%

\begin{document}

\title{Introducci�n pr�ctica a las tecnolog�as web}
\subtitle{Protagonizada por Mozilla WebMaker}
\author{Jes�s M. Gonz�lez Barahona}
\institute{jgb@gsyc.es ~~~~~ \url{http://twitter.com/jgbarah} \\
GSyC/LibreSoft, Universidad Rey Juan Carlos}

\date{Jornadas de acogida a alumnos de ESO \\ Escuela de Telecomunicaci�n \\
  Universidad Rey Juan Carlos \\
  Fuenlabrada, 24 de marzo de 2015}

\frame{
\maketitle
\begin{center}
\includegraphics[width=6cm]{format/gsyc-urjc}
\end{center}
}


% Si el titulo o el autor se quieren acortar para los pies de p�gina
% se pueden redefinir aqu�:
%\title{Titulo corto}
%\author{Autores abreviado}


% LICENCIA DE REDISTRIBUCION DE LAS TRANSPAS
\frame{
~
\vspace{4cm}

\begin{flushright}
\copyright 2015 Jes�s M. Gonz�lez Barahona. \\

  Algunos derechos reservados. \\
  Este art�culo se distribuye bajo la licencia \\
  ``Reconocimiento-CompartirIgual 3.0 Espa�a'' \\
  de Creative Commons, \\
  disponible en \\
  {\small \url{http://creativecommons.org/licenses/by-sa/3.0/es/deed.es}}
\end{flushright}
}

%%---------------------------------------------------------------

\begin{frame}
\frametitle{Antes de empezar...}

{\large
Vamos a usar:

\begin{itemize}
\item Firefox 3x \\
  (pero otros navegadores modernos servir�an)
\item Herramientas de Mozilla Webmaker
\item Alguna otra cosilla
\end{itemize}
\vspace{.5cm}

\begin{flushright}
\url{https://www.mozilla.org/firefox/}
\url{https://webmaker.org/}
\end{flushright}
}

\end{frame}

%%---------------------------------------------------------------

\begin{frame}
\frametitle{�Qu� visitamos cuando vamos de visita?}

{\large
Firefox Lightbeam

\vspace{.5cm}

Visualizaci�n de los sitios con los que has conectado \\
aunque ni hayas oido hablar de ellos

\vspace{.5cm}

Muchos de ellos son sitios que est�n trazando tu actividad

\begin{flushright}
\url{https://webmaker.org/private-eye}
\end{flushright}
}

\end{frame}


%%---------------------------------------------------------------

\begin{frame}
\frametitle{�Qu� descargamos cuando vemos una p�gina?}


{\Large

Firefox Web Console

\vspace{.5cm}

Instalado con Mozilla Firefox

\vspace{.5cm}

Veamos el panel ``Network''...
}

\end{frame}


%%---------------------------------------------------------------

\begin{frame}
\frametitle{Cambia los sitios que visitas (bueno, casi...)}


{\Large

Mozilla X-Ray Goggles

\vspace{.5cm}

\begin{flushright}
\url{https://webmaker.org/goggles}
\end{flushright}

}

\end{frame}

%%---------------------------------------------------------------

\begin{frame}
\frametitle{Crea p�ginas web}


{\Large

Mozilla Thimble

\vspace{.5cm}

\begin{flushright}
\url{https://thimble.webmaker.org/}
\end{flushright}

}

\end{frame}

%%---------------------------------------------------------------

\begin{frame}
\frametitle{Crea apps para el m�vil}


{\Large

Mozilla AppMaker

\vspace{.5cm}

\begin{flushright}
\url{https://apps.webmaker.org/}
\end{flushright}

}

\end{frame}

%%---------------------------------------------------------------

\begin{frame}
\frametitle{Crea multimedia ``de verdad''}


{\Large

Mozilla PopcornMaker

\vspace{.5cm}

\begin{flushright}
\url{https://popcorn.webmaker.org/}
\end{flushright}

}

\end{frame}


\end{document}
