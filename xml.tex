% $Id$
%


\section{Introducción a XML}

%%---------------------------------------------------------------

\begin{frame}
\frametitle{Introducción}


\begin{itemize}
\item XML: Extensible Markup Language
\item Norma promovida por el W3C (1998)
\item Es un dialecto simplificado de SGML (Standarized General Markup
  Language)
\item Pretende ser razonablemente simple
\item Se está usando en varios nuevos estándares: MathML, SMIL,
  (Synchronized Multimedia Integration Language), DocBook/XML, 
  XHTML, RSS, etc.
\end{itemize}

\end{frame}

%%---------------------------------------------------------------

\begin{frame}[fragile]
\frametitle{Características}

\begin{itemize}
\item Lenguaje de marcado: etiquetas de componentes según su semántica
\item Puede especificarse que un documento ha de estar organizado de
  cierta forma
\item Estructura jerárquica
\item Sintaxis básica:
  \begin{itemize}
  \item Marcas: 
\begin{verbatim}
<p> ... </p> 
<p/>
\end{verbatim}
  \item Atributos: 
\begin{verbatim}
<article lang="es">
\end{verbatim}
  \end{itemize}
\end{itemize}

\end{frame}

%%---------------------------------------------------------------

\begin{frame}[fragile]
\frametitle{Fichero XML sencillo}

\begin{verbatim}
<?xml version="1.0" ?>

<document>
Este es el documento
</document>
\end{verbatim}

\end{frame}

%%---------------------------------------------------------------

\begin{frame}[fragile]
\frametitle{Ejemplo más complejo}

Estructura muy simplificada de DocBook:

\begin{verbatim}
<?xml version="1.0" ?>
<article>
  <artheader>
    <title>El lenguaje XML</title>
    <author>Tim Ray</author>
  </artheader>
  <sect1>
    <para>...</para>
  </sect1>
  <sect2>
    <para>...</para>
  </sect2>
</article>
\end{verbatim}

\end{frame}

%%---------------------------------------------------------------

\begin{frame}
\frametitle{Definición y validación}

\begin{itemize}
\item XML ofrece la posibilidad de definir lenguajes (también llamados vocabularios) de etiquetas
\item La definición de un lenguaje XML especifica:
  \begin{itemize}
  \item Elementos del lenguaje
  \item Anidamiento de los elementos
  \item Atributos para cada elemento
  \item Atributos por defecto, atributos obligatorios
  \end{itemize}
\item El objetivo es poder validar que un documento XML tiene la información que se espera y con la estructura correcta.
\end{itemize}

\end{frame}
%%---------------------------------------------------------------

\begin{frame}
\frametitle{Definición y validación}

\begin{itemize}
\item Sistemas de definir lenguajes XML
  \begin{itemize}
  \item DTD (Document Type Definition): Estándar clásico
  \item XML Schema: Nuevo estándar. Muy completo y complejo
  \item RelaxNG: Alternativa al nuevo estándar. Más sencillo
  \end{itemize}
\item Cada documento XML debe tener una referencia a su definición al principio del fichero (si tiene una)
\end{itemize}

\end{frame}

%%---------------------------------------------------------------

\begin{frame}[fragile]
\frametitle{Ejemplos de lenguajes}

DocBook XML:

\begin{verbatim}
<?xml version="1.0" encoding="ISO-8859-1" ?>
<!DOCTYPE article PUBLIC 
   "-//Norman Walsh//DTD DocBk XML V3.1//EN"
   "/usr/lib/sgml/dtd/docbook-xml/docbookx.dtd">
\end{verbatim}

LaEspiral Document:

\begin{verbatim}
<?xml version="1.0" encoding="ISO-8859-1" ?>
<!DOCTYPE article PUBLIC 
   "-//laespiral.org//DTD LE-document 1.1//EN"
   "http://.../xml/styles/LE-document-1.1.dtd">
\end{verbatim}

\end{frame}

%%---------------------------------------------------------------

\begin{frame}[fragile]
\frametitle{Especificación (ejemplo)}

\begin{verbatim}
<?xml version="1.0" encoding="ISO-8859-1" ?>
<!DOCTYPE joke [
   <!ELEMENT joke (start, end+)>
   <!ATTLIST joke title CDATA #IMPLIED>
   <!ELEMENT start (#PCDATA)>
   <!ELEMENT end (#PCDATA)>
  ]>

<joke title="Van dos''>
  <start>Van dos por una calle</start>
  <end>y se cae el del medio</end>
  <end>y luego vienen</end>
</joke>
\end{verbatim}

\end{frame}

%%---------------------------------------------------------------

\begin{frame}
\frametitle{Bibliotecas para proceso XML}

Tipos de bibliotecas disponibles:

\begin{itemize}
\item DOM
  \begin{itemize}
  \item Document Object Model
  \item Lee todo el documento, y permite recorrer el árbol de
    elementos
  \item Permite modificar el documento XML
  \end{itemize}
\item SAX
  \begin{itemize}
  \item Standard API for XML
  \item Procesa el documento según está disponible
  \item Es my rápido y requiere menos memoria, pero es sólo para lectura
  \end{itemize}
\item Otras: variantes de DOM y SAX más sencillas pero no estándar
\end{itemize}

\end{frame}

%%---------------------------------------------------------------

\begin{frame}
\frametitle{Usos típicos de XML en aplicaciones web}

Gestión de interfaz de usuario en navegador:

\begin{itemize}
\item Manipulación de árbol DOM de la página HTML
\item Realizada normalmente mediante JavaScript
\item Interfaces de usuario mucho más ricas
\item Normalmente: combinación con código de aplicación web
\end{itemize}

Intercambio de datos entre aplicaciones web:

\begin{itemize}
\item Fuente para distintos formatos de documento \\
  (ej: documentos en Docbook servidos en HTML, ePub, etc.)
\item Canales (feeds) de una web
\item Interfaz para bases de datos (XPath, etc.)
\item Web semántica (RDF, RDFa)
\end{itemize}
\end{frame}

%%---------------------------------------------------------------

\begin{frame}
\frametitle{DOM de documento HTML en JavaScript (1)}

\begin{itemize}
\item node.parentNode: nodo padre
\item node.childNodes[1]: segundo nodo hijo
\item node.firstChild, x.lastChild: primer, último nodo hijo
\item document.getElementsByTagName(tag): \\
  vector con todos los elementos con etiqueta tag \\
  $<tag>XXX</tag>$
\item document.getElementById(id): \\
  elemento con identificador id \\
  $<tag id=''id''>XXX</tag>$
\end{itemize}
\end{frame}

%%---------------------------------------------------------------

\begin{frame}
\frametitle{DOM de documento HTML en JavaScript (2)}

\begin{itemize}
\item x.nodeValue: valor del nodo
\item x.innerHTML: HTML que ``cuelga'' del nodo
\item node.setAttribute('attr', val): \\
  asignar a atributo 'attr' a valor val
\item document.createElement(tag) \\
  crear nodo tag
\item document.createTextNode \\
  crear nodo con texto
\item node.appendChild(newnode) \\
  añade newnode como hijo de node
\item node.removeChild(remnode) \\
  elimina hijo remnode de node
\end{itemize}

\end{frame}

%%---------------------------------------------------------------

\begin{frame}[fragile]
\frametitle{DOM de documento HTML en JavaScript: ejemplo (1)}

\begin{verbatim}
<html>
  <head>
    <script type="text/javascript" charset="utf-8">
      function change()
      {
      document.getElementById('another').innerHTML =
         'Changed!';
      }
    </script>
  </head>
\end{verbatim}
\end{frame}

%%---------------------------------------------------------------

\begin{frame}[fragile]
\frametitle{DOM de documento HTML en JavaScript: ejemplo (2)}

\begin{verbatim}
<body>
  <h1>Simple DOM demo</h1>
  <p>First sentence</p>
  <p id="sentence">Second sentence, with id "sentence"</p>
  <p id="another">Another sentence, with id "another"</p>
  <p>Text of the sentence with id "sentence":
    <script type="text/javascript" charset="utf-8">
      document.write(
        document.getElementById('sentence').innerHTML
        );
    </script>
  </p>
\end{verbatim}
\end{frame}

%%---------------------------------------------------------------

\begin{frame}[fragile]
\frametitle{DOM de documento HTML en JavaScript: ejemplo (3)}

\begin{verbatim}
    <p>Text of the second sentence:
      <script type="text/javascript" charset="utf-8">
        document.write(
          document.getElementsByTagName('p')[1].innerHTML
          );
      </script>
    </p>
    <form>
      <input type="button" onclick="change()"
        value="Change" />
    </form>
  </body>
</html>
\end{verbatim}
\end{frame}

%%---------------------------------------------------------------

\begin{frame}
\frametitle{Preparación de documentos}

\begin{itemize}
\item Los documentos se escriben en un lenguaje XML (por ejemplo 
   Docbook)
\item Usando hojas de estilo XSL (Extensible Style Language) 
  pueden obtenerse diversos formatos
\item Ventajas
  \begin{itemize}
  \item Ayuda en la estructuración del texto
  \item Simplifica el tratamiento automático
  \item Simplifica la validación
  \item Se separa el contenido de la presentación
  \end{itemize}
\end{itemize}

\end{frame}

%%---------------------------------------------------------------

\begin{frame}
\frametitle{Canales (feeds) de una web}

\begin{itemize}
\item Introducido por Netscape en 1999, para definir ``canales'' (feeds)
\item Permite a los sitios web publicar una lista de las novedades del sitio
\item Esta lista es accedida por aplicaciones lectores de feeds
\item Los usuarios pueden subscribirse a los feeds y recibir avisos de los cambios
\item Lenguajes XML empleados
  \begin{itemize}
  \item RSS 0.91 (Rich Site Summary)
  \item RSS 0.9 y 1.0 (RDF Site Summary)
  \item RSS 2.0 (Really Simple Syndication)
  \item Atom 1.0
  \end{itemize}
\end{itemize}

\end{frame}

%%---------------------------------------------------------------

\begin{frame}[fragile]
\frametitle{Ejemplo de canal RSS (1)}

\begin{verbatim}
<?xml version="1.0" encoding="UTF-8"?>
<rss version="2.0">
<channel>
  <title>Menéame: publicadas</title>
	
  <link>http://meneame.net</link>
  <image>
    <title>Menéame: publicadas</title>
    <link>http://meneame.net</link>
    <url>http://meneame.net/img/mnm/eli-rss.png</url>
  </image>
  <description>Sitio colaborativo de publicación y comunicación entre 
   blogs</description>
\end{verbatim}

\end{frame}

%%---------------------------------------------------------------

\begin{frame}[fragile]
\frametitle{Ejemplo de canal RSS (2)}

\begin{verbatim}
  <pubDate>Sat, 22 Nov 2008 13:45:02 +0000</pubDate>
  <item>
    <title>¿Qué pasa cuando un familiar te pide que le montes un 
      PC?</title>
    <link>http://feedproxy.google.com/...-montes-pc</link>
    <pubDate>Sat, 22 Nov 2008 12:55:03 +0000</pubDate>
    <description>&lt;p&gt;Un breve relato que viene a ...
    </description>
  </item>
  <item>...</item>
  <item>...</item>
</channel>
\end{verbatim}

\end{frame}

%%---------------------------------------------------------------

\begin{frame}
\frametitle{Atom}

\begin{itemize}
\item Creado como evolución de RSS 2.0
\item Mejoras que introduce:
  \begin{itemize}
  \item Identificación del tipo de contenido de los elementos summary y content
  \item Incorpora XML namespaces para permitir extender Atom con otros vocabularios XML
  \end{itemize}
\item Mucho más apropiado para su interpretación por parte de programas:
  \begin{itemize}
  \item Esto está provocando su uso habitual en servicios web RESTful
  \end{itemize}
\end{itemize}
\end{frame}

%%---------------------------------------------------------------

\begin{frame}[fragile]
\frametitle{Ejemplo de canal Atom (1)}

\begin{verbatim}
<?xml version="1.0" encoding="utf-8"?>
<feed xmlns="http://www.w3.org/2005/Atom">
    <title>Ejemplo</title>
    <subtitle>Muy interesante</subtitle>
    <link href="http://example.org/"/>
    <updated>2003-12-13T18:30:02Z</updated>
    <author>
       <name>John Doe</name>
       <email>johndoe@example.com</email>
    </author>
    <id>urn:uuid:60a76c80-d399-11d9-b93C-0003939e0af6</id>
\end{verbatim}

\end{frame}

%%---------------------------------------------------------------

\begin{frame}[fragile]
\frametitle{Ejemplo de canal Atom (2)}

\begin{verbatim}
    <entry>
       <title>Atom-Powered Robots Run Amok</title>
       <link href="http://example.org/2003/12/13/atom03"/>
       <id>urn:uuid:1225c695-cfb8-4ebb-aaaa-80da344efa6a</id>
       <updated>2003-12-13T18:30:02Z</updated>
       <summary type="text">Some text.</summary>
    </entry>
</feed>
\end{verbatim}

\end{frame}

%%---------------------------------------------------------------

\begin{frame}[fragile]
\frametitle{Tipos de contenido en Atom}

\begin{verbatim}
 <content type="text">
    Contenido del feed
 </content>
 <content type="html">
    Contenido del &lt;em&gt;feed&lt;/em&gt;
 </content>
 <content type="xhtml" xmlns:xhtml="http://www.w3.org/1999/xhtml">
    <xhtml:div>
      Contenido del <xhtml:em>feed</xhtml:em>
    </xhtml:div>
 </content>
\end{verbatim}

\end{frame}

%%---------------------------------------------------------------

\begin{frame}[fragile]
\frametitle{Web semántica: RDF}

\begin{itemize}
\item RDF: Resource Description Framework
  \begin{itemize}
  \item Mecanismo uniforme para especificar la semántica de un recurso
    web, y su relación con otros.
  \item Cada documento RDF es una serie de tripletes.
  \end{itemize}
\item Modelo de metadatos basado en declaraciones sobre recursos
\item Sujeto (recurso), predicado (relación o aserción entre sujeto y
  objeto) y objeto (o valor)
\item Una colección de declaraciones RDF constituyen un grafo
  orientado y etiquetado
\item Es habitual que el recurso sea un URI, el objeto una cadena
  Unicode, y el predicado un URI
\end{itemize}

\end{frame}

%%---------------------------------------------------------------

\begin{frame}
\frametitle{Otras tecnologías de la familia XML}

\begin{itemize}
\item XLink: enlaces entre documentos XML
\item XSLT: conversión automática de XML a otros formatos
\item XPath: localización dentro de un documento (``query'' sobre XML)

\end{itemize}
\end{frame}

%%---------------------------------------------------------------

\begin{frame}
\frametitle{XML Namespaces}

\begin{itemize}
\item Permiten agrupar varios vocabularios XML en un único documento
\item Resuelven el problema de colisiones entre etiquetas con el mismo nombre
\end{itemize}

\end{frame}

%%---------------------------------------------------------------

\begin{frame}
\frametitle{Ejemplo: Problema}

Introducir en el canal de una web información geográfica
\begin{itemize}
\item Existen dos lenguajes XML distintos para cada información:
  \begin{itemize}
  \item RSS o Atom: Permiten dar información del canal y sus items
  \item geoRSS: Permite especificar longitud y latitud
  \end{itemize}
\end{itemize}

\end{frame}

%%---------------------------------------------------------------

\begin{frame}[fragile]
\frametitle{Ejemplo: Solución}

\begin{verbatim}
<rss xmlns:georss="http://www.georss.org/georss" 
     version="2.0">
<channel>
  <item>
    <title>En Málaga colocar publicidad en los parabrisas de
     los coches se multará con hasta 750 euros</title>
     <georss:point>36.7196745 -4.4200359</georss:point>
     ...
\end{verbatim}

\end{frame}

%%---------------------------------------------------------------

\begin{frame}
\frametitle{Uso de Namespaces en la práctica}

\begin{itemize}
\item El parser empleado debe soportar namespaces: la mayoría lo hacen ya
\item Problema: el mecanismo de validación original de XML, los DTDs, no 
  proporciona ninguna solución para validar documentos XML con namespaces
  \begin{itemize}
  \item Solución: usar XML Schema o RelaxNG
  \end{itemize}
\end{itemize}
\end{frame}


%%---------------------------------------------------------------

\begin{frame}
\frametitle{Alternativa a XML: JSON}

\begin{itemize}
\item JavaScript Object Notation
\item Definido en RFC 4627
\item Internet media type: application/json.
\item Permite representar estructuras de datos
\item Específicamente pensado para aplicaciones AJAX
\item Es un subconjunto de JavaScript (y de Python)
\end{itemize}
\end{frame}

%%---------------------------------------------------------------

\begin{frame}[fragile]
\frametitle{Ejemplo de JSON}

\begin{verbatim}
{
    "firstName": "John",
    "lastName": "Smith",
    "address": {
        "streetAddress": "21 2nd Street",
        "city": "New York",
        "state": "NY",
        "postalCode": 10021
    },
    "phoneNumbers": [
        "212 732-1234",
        "646 123-4567"
    ]
}
\end{verbatim}

\end{frame}

%%---------------------------------------------------------------

\begin{frame}
\frametitle{Uso en la práctica de JSON}

Existen intérpretes para casi cualquier lenguaje:
\begin{itemize}
\item JavaScript: La función eval() es capaz de intérpretarlo directamente. \\
  Pero mejor usar el módulo JSON
\item Python: módulo json
\item Java: JSON Tools, xstream, Restlet, ...
\item   ...
\end{itemize}

Extensiones:
\begin{itemize}
\item JSONT: Equivalente a XSLT para JSON
\item JSONP: convierte el contenido en activo al incluir la invocación a una función JavaScript.
\end{itemize}
\end{frame}


%%---------------------------------------------------------------

\begin{frame}
\frametitle{Ventajas e Inconvenientes}

Ventajas
\begin{itemize}
\item Requiere menos caractereres para la misma información (consume menos ancho de banda)
\item Es fácil escribir un intérprete rápido (incluso en un navegador)
\item Fácil de leer por personas
\end{itemize}

Inconvenientes
\begin{itemize}
\item No tiene mecanismos de definición de lenguajes y validación
\item No hay equivalentes a Atom, XSLT, ...
\end{itemize}
\end{frame}

  
%%---------------------------------------------------------------

\begin{frame}
\frametitle{Referencias}

\begin{itemize}
\item Introducción al XML, por Jaime Villate: \\
  \url{http://gsyc.escet.urjc.es/actividades/linuxprog/xml/xml-notas.html}
\item Especificación de XML: \\
  \url{http://www.w3.org/TR/xml-spec.html}
\item Especificación anotada de XML: \\
  \url{http://www.xml.com/xml/pub/axml/axmlintro.html}
\item Python and XML Processing: \\
  \url{http://pyxml.sourceforge.net/topics/}
\item SIG for XML Processing in Python: \\
  \url{http://www.python.org/sigs/xml-sig/}
\end{itemize}
\end{frame}

%%---------------------------------------------------------------

\begin{frame}
\frametitle{Referencias II}

\begin{itemize}
\item Python/XML HOWTO: \\
  \url{http://py-howto.sourceforge.net/xml-howto/}
\item XML Linking Language (XLink): \\
  \url{http://www.w3.org/TR/xlink/}
\item XML Path Language (XPath): \\
  \url{http://www.w3.org/TR/xpath}
\item XPath Tutorial, por Miloslav Nic y Jiri Jirat: \\
  \url{http://www.zvon.org/xxl/XPathTutorial/General/examples.html}
\item XML Schema: \\
  \url{http://www.w3.org/XML/Schema}
\end{itemize}
\end{frame}

%%---------------------------------------------------------------

\begin{frame}
\frametitle{Referencias III}

\begin{itemize}
\item Using W3C XML Schema, por  Eric van der Vlist: \\
  \url{http://www.xml.com/pub/a/2000/11/29/schemas/part1.html}
\item XSL Transformations (XSLT): \\
  \url{http://www.w3.org/TR/xslt}
\item Página DOM del W3C: \\
  \url{http://www.w3.org/DOM/}
\item Página de SAX:
  \url{http://www.megginson.com/SAX/}
\item Página de RDF:
  \url{http://www.w3.org/RDF/}
\item RSS 1.0: \\
  \url{http://groups.yahoo.com/group/rss-dev/files/specification.html}
\end{itemize}
\end{frame}

%%---------------------------------------------------------------

\begin{frame}
\frametitle{Referencias IV}

\begin{itemize}
\item XML Namespaces: \\
  \url{http://www.w3.org/TR/REC-xml-names/}
\item XML Schema: \\
  \url{http://www.w3.org/XML/Schema}
\item Relax NG: \\
  \url{http://relaxng.org}
\item Atom (RFC4287): \\
  \url{http://tools.ietf.org/html/rfc4287}
\item W3C DOM -Introduction: \\
  \url{http://www.quirksmode.org/dom/intro.html}
\item JSON: \\
  \url{http://json.org/}
\end{itemize}
\end{frame}
