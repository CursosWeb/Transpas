% $Id$
%


\section{Arquitectura modelo-vista-controlador}

%%---------------------------------------------------------------

\begin{frame}
\frametitle{�Qu� es la arquitectura MVC?}


\begin{itemize}
\item Patr�n de arquitectura que separa la l�gica de aplicaci�n de la interfaz de usuario
\item Modelo: Conocimiento (datos).
\item Vista: Representaci�n del modelo (filtrado, actualizaci�n)
\item Controlador: Enlace entre el usuario y la aplicaci�n. Permite que las vistas sean presentadas al usuario.
\item Ejemplo: HTML (modelo), CSS (vista), navegador (controlador)
\end{itemize}

\end{frame}

%%---------------------------------------------------------------

\begin{frame}
\frametitle{MVC en aplicaciones web (1)}

\begin{quotation}
The model is any of the logic or the database or any of the data itself. The view is simply how you lay the data out, how it is displayed. [...]

The controller in a web app is a bit more complicated, because it has two parts. The first part is the web server (such as a servlet container) that maps incoming HTTP URL requests to a particular handler for that request. The second part is those handlers themselves, which are in fact often called ``controllers''.[...]
\end{quotation}

\begin{flushright}
``The Importance of Model-View Separation'', Terence Parr \\
  (contin�a)
\end{flushright}

\end{frame}

%%---------------------------------------------------------------

\begin{frame}
\frametitle{MVC en aplicaciones web (2)}


\begin{itemize}
\item Modelo: \\ Descripci�n de la base de datos
\item Vista: \\ Interfaz de usuario (HTML) que presenta el modelo al usuario
\item Controlador: \\ Recibe indicaciones del usuario, indica cambios al modelo, elige vista para mostrar resultados
\end{itemize}

\end{frame}

%%---------------------------------------------------------------

\begin{frame}
\frametitle{MVC en Django}

\begin{itemize}
\item Vista: Qu� data se muestra al usuario (no c�mo se muestra) \\
  funciones en views.py
\item Vistas delegan la presentaci�n en plantillas (templates)
\item Modelo: Descripci�n de los datos \\
  clases en models.py
\item Controlador: maquinaria de Django \\
  incluyendo urls.py
\item Django como ``plataforma MTV'': \\
  modelo, plantilla (template), vista
\end{itemize}

\end{frame}



%%---------------------------------------------------------------

\begin{frame}
\frametitle{Referencias}

\begin{itemize}
\item MVC, Xerox PARC 1978-79: \\
  \url{http://heim.ifi.uio.no/~trygver/themes/mvc/mvc-index.html}
\item ``The Importance of Model-View Separation. A Conversation with Terence Parr'', by Bill Venners \\
  \url{http://www.artima.com/lejava/articles/stringtemplate.html}
\end{itemize}
\end{frame}
