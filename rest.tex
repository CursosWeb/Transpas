% $Id$
%

%%---------------------------------------------------------------
%%---------------------------------------------------------------
\section{REST: Representational State Transfer}

%%---------------------------------------------------------------
%\begin{frame}
%\frametitle{Servicios Web}

%\begin{itemize}
%\item Servicio web (\emph{web service}): \\
%      conjunto de protocolos y estándares 
%      que sirven para intercambiar datos entre aplicaciones.

%\item Enfoques sobre servicios web:

%\begin{itemize}
%\item Tradicional: Basado en estándares como SOAP, WS-*, ...
%\item Informal (POX, STRESTful): Arquitectura ad-hoc usando XML, JSON, ...
%\item RESTful: siguen las restricciones del estilo arquitectural de la web. 
%   Emplean XML, JSON y cualquier tipo MIME que sea útil
%\end{itemize}
%\end{itemize}

%\end{frame}


%%---------------------------------------------------------------
\begin{frame}
\frametitle{REST: El estilo arquitectural de la web}

\begin{itemize}
\item REST: REpresentational State Transfer

  \begin{quote}
  ``Conjunto de recursos web (máquina de estados), \\
  en el que se realizan acciones seleccionando recursos \\
  a los que se les aplican operaciones (transiciones de estado), \\
  obteniendo representaciones de ese estado \\
  (que recibe el navegador).''
  \end{quote}
  
\item Estilo arquitectural para sistemas distribuidos
  \begin{itemize}
  \item Ampliamente extendido en la web estática (ficheros y directorios)
  \item Parcialmente extendido en la web programática (aplicaciones y servicios web)
  \end{itemize}
\end{itemize}

\end{frame}

%%---------------------------------------------------------------

%%---------------------------------------------------------------
\begin{frame}
\frametitle{Principios REST}

\begin{itemize}
\item Cliente (interfaz de usuario) - servidor (lógica de negocio, datos)
\item Sin estado: cada petición lleva todo su contexto
\item Cacheable: cada respuesta es marcada como cacheable, o no
\item Interfaz uniforme (recurso, urls, métodos, representaciones...)
\item Sistema por capas: cada componente sólo ``ve'' la capa que invoca
\item Código en el cliente (sobre todo, para interfaz de usuario)
\end{itemize}

\end{frame}

%%---------------------------------------------------------------
\begin{frame}
\frametitle{Interfaz: Cada recurso debe tener una URL}

Un recurso puede ser cualquier tipo de información que quiere hacerse visible en la web: documentos, imágenes, servicios, gente

~

\url{http://example.com/profiles}

\url{http://example.com/profiles/juan}

\url{http://example.com/posts/2007/10/11/ventajas_rest}

\url{http://example.com/posts?f=2007/10/11&t=2007/11/17}

\url{http://example.com/shop/4554/}

\url{http://example.com/shop/4554/products}

\url{http://example.com/shop/4554/products/15}

\url{http://example.com/orders/juan/15}

\end{frame}
%%---------------------------------------------------------------

\begin{frame}
\frametitle{Interfaz: Cada recurso debe tener una URL}

Dos tipos de recursos:

\begin{itemize}
  \item Colecciones, como http://example.com/resources/
  \item Elementos, como http://example.com/resources/142
\end{itemize}

Los métodos tienen un significado ligeramente diferente se trate de
una colección o un elemento.

\end{frame}


%%---------------------------------------------------------------
\begin{frame}[fragile]
\frametitle{Interfaz: Los recursos tienen hiperenlaces a otros recursos}

Los formatos más usados (XHTML, Atom, ...) ya lo permiten.
También aplica a los formatos inventados por nosotros mismos

\begin{verbatim}
<profiles self="http://example.com/profiles">
 <profile type="moderator" 
       self="http://example.com/profiles/123">
  <name>Pepito Pérez</name>
  <company ref="http://example.com/companies/321">
    Compañía XYZ</company>
  </profile>
...
</profiles>
\end{verbatim}

\end{frame}
%%---------------------------------------------------------------


%%---------------------------------------------------------------
\begin{frame}
\frametitle{Interfaz: Conjunto limitado y estándar de métodos}

\begin{itemize}
\item GET:  Obtener información (quizá de caché)
  \begin{itemize}
  \item No cambia estado, idempotente
  \end{itemize}
\item PUT: Actualizar o crear un recurso conocida su URL 
  \begin{itemize}
  \item Cambia estado, idempotente
  \end{itemize}
\item POST: Crear un recurso conociendo la URL de un constructor de recursos
  \begin{itemize}
  \item Cambia estado, NO idempotente
  \end{itemize}
\item DELETE: Borrar un recurso conocida su URL
  \begin{itemize}
  \item Cambia estado, idempotente
  \end{itemize}
\end{itemize}

\end{frame}
%%---------------------------------------------------------------


%%---------------------------------------------------------------
\begin{frame}[fragile]
\frametitle{Interfaz: Múltiples representaciones por recurso}

Las representaciones muestran el estado actual del recurso en un formato dado

\begin{verbatim}
GET /profile/1234
Host: example.com
Accept: text/html

GET /profile/1234
Host: example.com
Accept: text/x-vcard

GET /profile/1234
Host: example.com
Accept: application/mi-vocabulario-propio+xml
\end{verbatim}


\end{frame}
%%---------------------------------------------------------------


%%---------------------------------------------------------------
\begin{frame}
\frametitle{Interfaz: Comunicación sin estado}

\begin{itemize}
\item En REST se mantiene estado, pero sólo en los recursos, no en la aplicación (o sesión). 

\item Una petición del cliente al servidor debe contener todo el contexto necesario (no puede basarse en información previa asociada a la comunicación). 

\item No hay datos que permitan identificar sesión en el servidor. El servidor sólo guarda y maneja el estado de los recursos que aloja. 

\item El cliente puede guardar identificación de sesión (y enviárselo al servidor), por ejemplo en forma de cookies.

\item Esto permite escalabilidad (servidores más sencillos). Implica mayor ancho de banda, pero facilita las \emph{cachés}.
\end{itemize}

\end{frame}
%%---------------------------------------------------------------


%%---------------------------------------------------------------
\begin{frame}
\frametitle{REST: Ventajas de REST}

\begin{itemize}
\item Maximiza la reutilización
  \begin{itemize}
  \item Todos los recursos tienen identificadores = hacen más grande la Web
  \item La visibilidad de los recursos que forman una aplicación/servicio permite usos no previstos
  \end{itemize}
\item Minimiza el acoplamiento y permite la evolución
  \begin{itemize}
  \item La interfaz uniforme esconde detalles de implementación
  \item El uso de hipertexto permite que sólo la primera URL usada para acceder acople un sistema a otro
  \end{itemize}
\item Elimina condiciones de fallo parcial
  \begin{itemize}
  \item Fallo del servidor no afecta al cliente
  \item El estado siempre puede ser recuperado como un recurso
  \end{itemize}
\item Escalado sin límites
  \begin{itemize}
  \item Los servicios pueden ser replicados en cluster, cacheados y ayudados por sistemas intermediarios
  \end{itemize}
\item Unifica los mundos de las aplicaciones web y servicios web. El mismo código puede servir para los dos fines.
\end{itemize}

\end{frame}
%%---------------------------------------------------------------

%%---------------------------------------------------------------
\begin{frame}
\frametitle{Funcionalidades de HTTP no tan conocidas}

\begin{itemize}
\item Códigos de respuesta estandarizados: los entiende un navegador, los proxies, o nuestras propias aplicaciones
  \begin{itemize}
  \item Information: 1xx, Success 2xx, Redirection 3xx, Client Error 4xx, Server Error 5xx
  \end{itemize}
\item Negociación de contenido: la misma URL puede mandar la representación más adecuada al cliente (HTML, Atom, texto)
\item Redirecciones
\item Cacheado (incluyendo mecanismos para especificar el periodo de validez y expiraciones)
  \begin{itemize}
  \item Dos tipos: browser, proxy
  \item Cabecera HTTP: Cache-Control
  \end{itemize}
\end{itemize}

\end{frame}
%%---------------------------------------------------------------

%%---------------------------------------------------------------
\begin{frame}
\frametitle{Funcionalidades de HTTP no tan conocidas (2)}

\begin{itemize}
\item Compresión
\item División de la respuesta en partes (Chunking)
\item GET condicional (sólo se envía la respuesta si ha habido cambios)
  \begin{itemize}
  \item Permite ahorrar ancho de banda, procesado en el cliente y (posiblemente) procesado en el servidor
  \item Dos mecanismos:
    \begin{itemize}
    \item ETag y If-None-Match: identificador asociado al estado de un recurso. Ejemplo: hash de la representación en un formato dado
    \item Last-Modifed y If-Modified-Since: fecha de última actualización
    \end{itemize}
  \end{itemize}
\end{itemize}

\end{frame}
%%---------------------------------------------------------------


%%---------------------------------------------------------------
\begin{frame}[fragile]
\frametitle{Ejemplo de interacción HTTP: Petición}

\begin{verbatim}
GET / HTTP/1.1
Host: www.ejemplo.com
User-Agent: Mozilla/5.0 (Windows; U; Windows NT 5.1; 
   en-US; rv:1.8.1.3)...
Accept: text/xml,application/xml,application/xhtml+xml,
   text/html;q=0.9...
Accept-Language: en-us,en;q=0.5
Accept-Encoding: gzip,deflate
Accept-Charset: ISO-8859-1,utf-8;q=0.7,*;q=0.7
If-None-Match: "4c083-268-423f1dc5"
If-Modified-Since: "4c083-268-423f1dc5"
Keep-Alive: 300
Connection: keep-alive
Cookie: [...]
Cache-Control: max-age=0
\end{verbatim}

\end{frame}
%%---------------------------------------------------------------


%%---------------------------------------------------------------
\begin{frame}[fragile]
\frametitle{Ejemplo de interacción HTTP: Respuesta}

\begin{verbatim}
HTTP/1.x 200 OK
Date: Thu, 17 May 2007 14:06:30 GMT
Server: Microsoft-IIS/6.0
ETag: "4c083-268-423f1dc6"
Last-Modified: Mon, 21 Mar 2005 19:17:26 GMT
Cache-Control: private
Content-Type: text/html; charset=utf-8
Content-Length: 16850
\end{verbatim}

\end{frame}
%%---------------------------------------------------------------

%%---------------------------------------------------------------
\begin{frame}
\frametitle{Mecanismos de seguridad}

\begin{itemize}
\item Autenticación:
  \begin{itemize}
  \item HTTP Basic
  \item HTTP Digest
  \item OAuth
  \end{itemize}
\item Importante: los mecanismos de autenticación de HTTP son extensibles
\item Cifrado y firma digital: SSL
\end{itemize}

\end{frame}
%%---------------------------------------------------------------


%%---------------------------------------------------------------
\begin{frame}
\frametitle{Límites actuales de REST en la web}

\begin{itemize}
\item Obliga a los desarrolladores a pensar diferente
\item Carencia de soporte de PUT y DELETE en navegadores y cortado en cortafuegos
\item Mecanismos de seguridad limitados (en comparación con WS-Security): está mejorando rápidamente
\item La negociación de contenidos no funciona del todo bien en la práctica
\item Algunos piden un método más: PATCH
\item Poca documentación
\end{itemize}

\end{frame}
%%---------------------------------------------------------------

%%---------------------------------------------------------------
\begin{frame}
\frametitle{Ejemplos RESTful}

\begin{itemize}
\item En la web  
  \begin{itemize}
  \item Todos los sitios web estáticos
  \item Servicios web de sólo lectura
  \item Servicios web Google (GData)
  \item Aplicaciones web hechas con Ruby Rails 1.2+
  \end{itemize}
\item Estándares basados en REST
  \begin{itemize}
  \item Atom Publishing Protocol
  \item WebDAV
  \item Amazon S3
  \item GData (basado en Atom Publisihg Protocol)
  \item OpenSearch (basado en Atom Publisihg Protocol)
  \end{itemize}
\end{itemize}

\end{frame}
%%---------------------------------------------------------------

%%---------------------------------------------------------------
\begin{frame}
\frametitle{Conclusiones}

\begin{itemize}
\item REST es un estilo arquitectural para aplicaciones distribuidas
\item Introduce restricciones que producen propiedades deseables a cambio
  \begin{itemize}
  \item Estas propiedades han permitido la expansión con éxito de la web
  \item Las restricciones también pueden aplicarse a las aplicaciones y servicios web desarrollados para mantener las propiedades
  \end{itemize}
\item Los protocolos de la web, HTTP y URL, facilitan el desarrollo de arquitecturas RESTful y permiten explotar sus propiedades
\item REST no es el único estilo arquitectural
  \begin{itemize}
  \item Es posible eliminar restricciones si no nos importa perder propiedades
  \item Hay otros estilos arquitecturales con otras propiedades. Ejemplo: RPC
  \end{itemize}
\end{itemize}

\end{frame}
%%---------------------------------------------------------------


%%---------------------------------------------------------------

\begin{frame}
\frametitle{Referencias}

\begin{itemize}
\item ``Chapter 5: Representational State Transfer (REST)'', Fielding, Roy Thomas, en ``Architectural Styles and the Design of Network-based Software Architectures'' (PhD thesis) University of California, Irvine 2000.
\item ``RESTful Web Services'',  Leonard Richardson, Sam Ruby, O'Reilly Press
\item ``Architectural Styles and the Design of Network-based Software Architectures'' (Capítulo 5) \\
  {\footnotesize \url{http://www.ics.uci.edu/~fielding/pubs/dissertation/rest_arch_style.htm}}
\item ``Best Practices for Designing a Pragmatic RESTful API'', 
  Vinay Sahni \\
  {\footnotesize \url{http://www.vinaysahni.com/best-practices-for-a-pragmatic-restful-api}}
\item Atom Publishing Protocol: \\
  \url{http://tools.ietf.org/html/rfc5023}
\end{itemize}
\end{frame}

