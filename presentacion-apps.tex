% $Id$
%


\section{Presentación de la asignatura}

%%---------------------------------------------------------------

\begin{frame}
\frametitle{Datos, datos, datos}

\begin{itemize}
\item Profesores:
  \begin{itemize}
  \item Jesús M. González Barahona \\
    jgb @ gsyc.es, @jgbarah (Twitter)
  \item Gregorio Robles \\
    grex @ gsyc.es, @gregoriorobles (Twitter)
  \item Departamento de Sistemas Telemáticos y Computación (GSyC)
  \item Despachos: 003 Biblioteca y 110 Departamental III
  \end{itemize}
\item Hay examen de teoría
\item Hay examen de prácticas
\item Laboratorio 209, Laboratorios III
\end{itemize}

\begin{flushright}
Campus virtual: \\
{\small \url{http://campusvirtual.urjc.es/}} \\
Materiales: \\
{\small \url{http://cursosweb.github.io/}} \\
\end{flushright}

\end{frame}

%%---------------------------------------------------------------

\begin{frame}
\frametitle{¿De qué va esto?}


\begin{itemize}
\item HTML, HTML5
\item CSS, CSS3
\item JavaScript
\item SVG, video
\item Geolocalización
\item ...
\end{itemize}

Énfasis en:

\begin{itemize}
\item Cómo encajan las piezas
\item En la medida de lo posible, ``manos en la masa''.
\end{itemize}


\end{frame}


%%---------------------------------------------------------------

\begin{frame}
\frametitle{Ejemplos}

\vspace{1cm}

\begin{itemize}
  \item Utilizando Google Maps para ver las antípodas \\
    \url{http://www.antipodemap.com/}
  \item Pintando en el Canvas \\
    \url{http://www.williammalone.com/articles/create-html5-canvas-javascript-drawing-app/}
  \item Transiciones CSS3 \\
    \url{http://css3.bradshawenterprises.com/transitions/}
  \item Comecocos \\ 
    \url{http://www.canvasdemos.com/2010/07/30/pacman/}
  \item Prácticas del curso pasado...
\end{itemize}

\end{frame}

%%---------------------------------------------------------------

\begin{frame}
\frametitle{Metodología}

\vspace{1cm}

\begin{itemize}
\item Clases de teoría y de prácticas, pero...
\item Teoría en prácticas, prácticas en teoría
\item Uso de resolución de problemas para aprender
\item Fundamentalmente, entender lo fundamental
\item Objetivo principal: conceptos básicos de construcción de aplicaciones HTML5 portables
\end{itemize}

\end{frame}

%% %%---------------------------------------------------------------

%% \begin{frame}
%% \frametitle{Evaluación}

%% \begin{itemize}
%% \item Teoría (obligatorio): nota de 0 a 5.
%% \item Práctica final (obligatorio): nota de 0 a 2.
%% \item Opciones y mejoras práctica final: nota de 0 a 3
%% \item Prácticas incrementales: 0 a 1
%% \item Ejercicios en foro: nota de 0 a 1
%% \item Nota final: Suma de notas, moderada por la interpretación del profesor
%% \item Mínimo para aprobar:
%%       \begin{itemize}
%%       \item Aprobado en teoría (2.5) y práctica final (1), y
%%       \item 5 puntos de nota final
%%       \end{itemize}
%% \end{itemize}

%% \end{frame}

%% %%---------------------------------------------------------------

%% \begin{frame}
%% \frametitle{Evaluación (2)}

%% \begin{itemize}
%% \item Evaluación teoría: prueba escrita
%% \item Evaluación prácticas incrementales (evaluación continua):
%%       \begin{itemize}
%%       \item entre 0 y 1 (sobre todo las extensiones)
%%       \item es muy recomendable hacerlas
%%       \end{itemize}
%% \item Evaluación práctica final
%%       \begin{itemize}
%%       \item posibilidad de examen presencial para práctica final
%%       \item ¡tiene que funcionar en el laboratorio!
%%       \item enunciado mínimo obligatorio supone 1, se llega a 2 sólo con calidad y cuidado en los detalles
%%       \end{itemize}
%% \item Opciones y mejoras práctica final:
%%       \begin{itemize}
%%       \item permiten subir la nota mucho
%%       \end{itemize}
%% \item Evaluación ejercicios (evaluación continua):
%%       \begin{itemize}
%%       \item preguntas y ejercicios en foro
%%       \end{itemize}
%% \item Evaluación extraordinaria:
%%   \begin{itemize}
%%   \item prueba escrita (si no se aprobó la ordinaria)
%%   \item nueva práctica final (si no se aprobó la ordinaria)
%%   \end{itemize}
%% \end{itemize}

%% \end{frame}

%%---------------------------------------------------------------

\begin{frame}
\frametitle{¡Ánimo!}

\begin{center}
{\huge Aquí se enseñan cómo son las cosas \\
  que se usan en el mundo real \\
  ~ \\
  Las buenas noticias son... \\
  que no son tan difíciles\\}
\end{center}

\end{frame}

