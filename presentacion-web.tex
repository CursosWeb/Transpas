% $Id$
%


\section{Presentación de la asignatura}

%%---------------------------------------------------------------

\begin{frame}
\frametitle{Datos, datos, datos}

\begin{itemize}
\item Profesores:
  \begin{itemize}
  \item Jesús M. González Barahona (jgb @ gsyc.es)
  \item Gregorio Robles (grex @ gsyc.es)
  \item Departamento de Sistemas Telemáticos y Computación (GSyC)
  \item Despachos: 003 Biblioteca y 110 Departamental III
  \end{itemize}
\item Hay examen de teoría
\item Hay examen de prácticas
\item Laboratorio 209, Laboratorios III
\end{itemize}

\begin{flushright}
Campus virtual: \\
{\small \url{http://campusvirtual.urjc.es/}} \\
Materiales: \\
{\small \url{http://cursosweb.github.io/}} \\
\end{flushright}

\end{frame}

%%---------------------------------------------------------------

\begin{frame}
\frametitle{¿De qué va esto?}


\begin{itemize}
\item Cómo se construyen los sistemas reales que se usan en Inet
\item Qué tecnologías se están usando
\item Qué esquemas de seguridad hay
\item Cómo encajan las piezas
\item En la medida de lo posible, ``manos en la masa''.
\end{itemize}


\end{frame}


%%---------------------------------------------------------------

\begin{frame}
\frametitle{Ejemplos}

\vspace{1cm}

\begin{itemize}
\item ¿Qué es una aplicación web?
\item Cómo construir un servicio REST
\item Acaba con la magia de los servicios web
%\item ¿Sabes qué es IPSEC? ¿HTTPS? ¿Un certificado?
%\item ¿Has visto alguna vez una traza de streaming de video?
\item ¿Qué es el web 2.0?
\item ¿Cómo se hace un mashup?
\item ¿Cómo puedo interaccionar con los servicios más populares?
\item ¿Qué es HTML5?
\end{itemize}

\end{frame}

%%---------------------------------------------------------------

\begin{frame}
\frametitle{Metodología}

\vspace{1cm}

\begin{itemize}
\item Clases de teoría y de prácticas, pero...
\item Teoría en prácticas, prácticas en teoría
\item Uso de resolución de problemas para aprender
\item Fundamentalmente, entender lo fundamental
\item Objetivo principal: conceptos básicos de construcción de sitios
  web modernos
\end{itemize}

\end{frame}

%%---------------------------------------------------------------

\begin{frame}
\frametitle{Prácticas}

Ejemplos del pasado:

\begin{itemize}
\item Servicio de apoyo a la docencia
\item Sitio de intercambio de fotos
\item Aplicación web de autoevaluación docente
\item Agregador de blogs (canales RSS)
\item Agregador de microblogs (Identi.ca, Twitter)
\end{itemize}

Tecnología de apoyo:

\begin{itemize}
\item Python, \url{http://python.org}
\item Django, \url{http://www.djangoproject.com/}
\item Alguna más...
\end{itemize}

\end{frame}

%%---------------------------------------------------------------

\begin{frame}
\frametitle{Evaluación}

\begin{itemize}
\item Teoría (obligatorio): nota de 0 a 5.
\item Práctica final (obligatorio): nota de 0 a 2.
\item Opciones y mejoras práctica final: nota de 0 a 3
\item Prácticas incrementales: 0 a 1
\item Ejercicios en foro: nota de 0 a 1
\item Nota final: Suma de notas, moderada por la interpretación del profesor
\item Mínimo para aprobar:
      \begin{itemize}
      \item Aprobado en teoría (2.5) y práctica final (1), y
      \item 5 puntos de nota final
      \end{itemize}
\end{itemize}

\end{frame}

%%---------------------------------------------------------------

\begin{frame}
\frametitle{Evaluación (2)}

\begin{itemize}
\item Evaluación teoría: prueba escrita
\item Evaluación prácticas incrementales (evaluación continua):
      \begin{itemize}
      \item entre 0 y 1 (sobre todo las extensiones)
      \item es muy recomendable hacerlas
      \end{itemize}
\item Evaluación práctica final
      \begin{itemize}
      \item posibilidad de examen presencial para práctica final
      \item ¡tiene que funcionar en el laboratorio!
      \item enunciado mínimo obligatorio supone 1, se llega a 2 sólo con calidad y cuidado en los detalles
      \end{itemize}
\item Opciones y mejoras práctica final:
      \begin{itemize}
      \item permiten subir la nota mucho
      \end{itemize}
\item Evaluación ejercicios (evaluación continua):
      \begin{itemize}
      \item preguntas y ejercicios en foro
      \end{itemize}
\item Evaluación extraordinaria:
  \begin{itemize}
  \item prueba escrita (si no se aprobó la ordinaria)
  \item nueva práctica final (si no se aprobó la ordinaria)
  \end{itemize}
\end{itemize}

\end{frame}

%%---------------------------------------------------------------

\begin{frame}
\frametitle{¡Ánimo!}

\begin{center}
{\huge Aquí se enseñan cómo son las cosas \\
  que se usan en el mundo real \\
  ~ \\
  Las buenas noticias son... \\
  que no son tan difíciles\\}
\end{center}

\end{frame}


%%%%%%%%%%%%%%%%%%%%%%%%%%%%%%%%%%%%%%%%%%%%%%%%%%%%%%%%%%%%%%%%%%%%%%

%% \begin{frame}
%% \frametitle{Referencias (orientativas)}

%% \begin{itemize}
%% \item Philip Greenspun, \textsl{Software Engineering for Internet Applications}:\\
%%   \url{http://philip.greenspun.com/seia/} \\
%%   utilizado en un curso del MIT \\
%%   \url{http://philip.greenspun.com/teaching/one-term-web}

%% \item Transparencias de la asignatura

%% \item Materiales mencionados en cada capítulo

%% \end{itemize}


%% \end{frame}



%\end{document}
